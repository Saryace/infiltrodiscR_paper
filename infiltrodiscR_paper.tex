% Options for packages loaded elsewhere
\PassOptionsToPackage{unicode}{hyperref}
\PassOptionsToPackage{hyphens}{url}
\PassOptionsToPackage{dvipsnames,svgnames,x11names}{xcolor}
%
\documentclass[
]{article}

\usepackage{amsmath,amssymb}
\usepackage{lmodern}
\usepackage{iftex}
\ifPDFTeX
  \usepackage[T1]{fontenc}
  \usepackage[utf8]{inputenc}
  \usepackage{textcomp} % provide euro and other symbols
\else % if luatex or xetex
  \usepackage{unicode-math}
  \defaultfontfeatures{Scale=MatchLowercase}
  \defaultfontfeatures[\rmfamily]{Ligatures=TeX,Scale=1}
\fi
% Use upquote if available, for straight quotes in verbatim environments
\IfFileExists{upquote.sty}{\usepackage{upquote}}{}
\IfFileExists{microtype.sty}{% use microtype if available
  \usepackage[]{microtype}
  \UseMicrotypeSet[protrusion]{basicmath} % disable protrusion for tt fonts
}{}
\makeatletter
\@ifundefined{KOMAClassName}{% if non-KOMA class
  \IfFileExists{parskip.sty}{%
    \usepackage{parskip}
  }{% else
    \setlength{\parindent}{0pt}
    \setlength{\parskip}{6pt plus 2pt minus 1pt}}
}{% if KOMA class
  \KOMAoptions{parskip=half}}
\makeatother
\usepackage{xcolor}
\setlength{\emergencystretch}{3em} % prevent overfull lines
\setcounter{secnumdepth}{-\maxdimen} % remove section numbering
% Make \paragraph and \subparagraph free-standing
\ifx\paragraph\undefined\else
  \let\oldparagraph\paragraph
  \renewcommand{\paragraph}[1]{\oldparagraph{#1}\mbox{}}
\fi
\ifx\subparagraph\undefined\else
  \let\oldsubparagraph\subparagraph
  \renewcommand{\subparagraph}[1]{\oldsubparagraph{#1}\mbox{}}
\fi

\usepackage{color}
\usepackage{fancyvrb}
\newcommand{\VerbBar}{|}
\newcommand{\VERB}{\Verb[commandchars=\\\{\}]}
\DefineVerbatimEnvironment{Highlighting}{Verbatim}{commandchars=\\\{\}}
% Add ',fontsize=\small' for more characters per line
\usepackage{framed}
\definecolor{shadecolor}{RGB}{241,243,245}
\newenvironment{Shaded}{\begin{snugshade}}{\end{snugshade}}
\newcommand{\AlertTok}[1]{\textcolor[rgb]{0.68,0.00,0.00}{#1}}
\newcommand{\AnnotationTok}[1]{\textcolor[rgb]{0.37,0.37,0.37}{#1}}
\newcommand{\AttributeTok}[1]{\textcolor[rgb]{0.40,0.45,0.13}{#1}}
\newcommand{\BaseNTok}[1]{\textcolor[rgb]{0.68,0.00,0.00}{#1}}
\newcommand{\BuiltInTok}[1]{\textcolor[rgb]{0.00,0.23,0.31}{#1}}
\newcommand{\CharTok}[1]{\textcolor[rgb]{0.13,0.47,0.30}{#1}}
\newcommand{\CommentTok}[1]{\textcolor[rgb]{0.37,0.37,0.37}{#1}}
\newcommand{\CommentVarTok}[1]{\textcolor[rgb]{0.37,0.37,0.37}{\textit{#1}}}
\newcommand{\ConstantTok}[1]{\textcolor[rgb]{0.56,0.35,0.01}{#1}}
\newcommand{\ControlFlowTok}[1]{\textcolor[rgb]{0.00,0.23,0.31}{#1}}
\newcommand{\DataTypeTok}[1]{\textcolor[rgb]{0.68,0.00,0.00}{#1}}
\newcommand{\DecValTok}[1]{\textcolor[rgb]{0.68,0.00,0.00}{#1}}
\newcommand{\DocumentationTok}[1]{\textcolor[rgb]{0.37,0.37,0.37}{\textit{#1}}}
\newcommand{\ErrorTok}[1]{\textcolor[rgb]{0.68,0.00,0.00}{#1}}
\newcommand{\ExtensionTok}[1]{\textcolor[rgb]{0.00,0.23,0.31}{#1}}
\newcommand{\FloatTok}[1]{\textcolor[rgb]{0.68,0.00,0.00}{#1}}
\newcommand{\FunctionTok}[1]{\textcolor[rgb]{0.28,0.35,0.67}{#1}}
\newcommand{\ImportTok}[1]{\textcolor[rgb]{0.00,0.46,0.62}{#1}}
\newcommand{\InformationTok}[1]{\textcolor[rgb]{0.37,0.37,0.37}{#1}}
\newcommand{\KeywordTok}[1]{\textcolor[rgb]{0.00,0.23,0.31}{#1}}
\newcommand{\NormalTok}[1]{\textcolor[rgb]{0.00,0.23,0.31}{#1}}
\newcommand{\OperatorTok}[1]{\textcolor[rgb]{0.37,0.37,0.37}{#1}}
\newcommand{\OtherTok}[1]{\textcolor[rgb]{0.00,0.23,0.31}{#1}}
\newcommand{\PreprocessorTok}[1]{\textcolor[rgb]{0.68,0.00,0.00}{#1}}
\newcommand{\RegionMarkerTok}[1]{\textcolor[rgb]{0.00,0.23,0.31}{#1}}
\newcommand{\SpecialCharTok}[1]{\textcolor[rgb]{0.37,0.37,0.37}{#1}}
\newcommand{\SpecialStringTok}[1]{\textcolor[rgb]{0.13,0.47,0.30}{#1}}
\newcommand{\StringTok}[1]{\textcolor[rgb]{0.13,0.47,0.30}{#1}}
\newcommand{\VariableTok}[1]{\textcolor[rgb]{0.07,0.07,0.07}{#1}}
\newcommand{\VerbatimStringTok}[1]{\textcolor[rgb]{0.13,0.47,0.30}{#1}}
\newcommand{\WarningTok}[1]{\textcolor[rgb]{0.37,0.37,0.37}{\textit{#1}}}

\providecommand{\tightlist}{%
  \setlength{\itemsep}{0pt}\setlength{\parskip}{0pt}}\usepackage{longtable,booktabs,array}
\usepackage{calc} % for calculating minipage widths
% Correct order of tables after \paragraph or \subparagraph
\usepackage{etoolbox}
\makeatletter
\patchcmd\longtable{\par}{\if@noskipsec\mbox{}\fi\par}{}{}
\makeatother
% Allow footnotes in longtable head/foot
\IfFileExists{footnotehyper.sty}{\usepackage{footnotehyper}}{\usepackage{footnote}}
\makesavenoteenv{longtable}
\usepackage{graphicx}
\makeatletter
\def\maxwidth{\ifdim\Gin@nat@width>\linewidth\linewidth\else\Gin@nat@width\fi}
\def\maxheight{\ifdim\Gin@nat@height>\textheight\textheight\else\Gin@nat@height\fi}
\makeatother
% Scale images if necessary, so that they will not overflow the page
% margins by default, and it is still possible to overwrite the defaults
% using explicit options in \includegraphics[width, height, ...]{}
\setkeys{Gin}{width=\maxwidth,height=\maxheight,keepaspectratio}
% Set default figure placement to htbp
\makeatletter
\def\fps@figure{htbp}
\makeatother
\newlength{\cslhangindent}
\setlength{\cslhangindent}{1.5em}
\newlength{\csllabelwidth}
\setlength{\csllabelwidth}{3em}
\newlength{\cslentryspacingunit} % times entry-spacing
\setlength{\cslentryspacingunit}{\parskip}
\newenvironment{CSLReferences}[2] % #1 hanging-ident, #2 entry spacing
 {% don't indent paragraphs
  \setlength{\parindent}{0pt}
  % turn on hanging indent if param 1 is 1
  \ifodd #1
  \let\oldpar\par
  \def\par{\hangindent=\cslhangindent\oldpar}
  \fi
  % set entry spacing
  \setlength{\parskip}{#2\cslentryspacingunit}
 }%
 {}
\usepackage{calc}
\newcommand{\CSLBlock}[1]{#1\hfill\break}
\newcommand{\CSLLeftMargin}[1]{\parbox[t]{\csllabelwidth}{#1}}
\newcommand{\CSLRightInline}[1]{\parbox[t]{\linewidth - \csllabelwidth}{#1}\break}
\newcommand{\CSLIndent}[1]{\hspace{\cslhangindent}#1}

\usepackage{booktabs}
\usepackage{longtable}
\usepackage{array}
\usepackage{multirow}
\usepackage{wrapfig}
\usepackage{float}
\usepackage{colortbl}
\usepackage{pdflscape}
\usepackage{tabu}
\usepackage{threeparttable}
\usepackage{threeparttablex}
\usepackage[normalem]{ulem}
\usepackage{makecell}
\usepackage{xcolor}
\usepackage[noblocks]{authblk}
\renewcommand*{\Authsep}{, }
\renewcommand*{\Authand}{, }
\renewcommand*{\Authands}{, }
\renewcommand\Affilfont{\small}
\makeatletter
\makeatother
\makeatletter
\makeatother
\makeatletter
\@ifpackageloaded{caption}{}{\usepackage{caption}}
\AtBeginDocument{%
\ifdefined\contentsname
  \renewcommand*\contentsname{Table of contents}
\else
  \newcommand\contentsname{Table of contents}
\fi
\ifdefined\listfigurename
  \renewcommand*\listfigurename{List of Figures}
\else
  \newcommand\listfigurename{List of Figures}
\fi
\ifdefined\listtablename
  \renewcommand*\listtablename{List of Tables}
\else
  \newcommand\listtablename{List of Tables}
\fi
\ifdefined\figurename
  \renewcommand*\figurename{Figure}
\else
  \newcommand\figurename{Figure}
\fi
\ifdefined\tablename
  \renewcommand*\tablename{Table}
\else
  \newcommand\tablename{Table}
\fi
}
\@ifpackageloaded{float}{}{\usepackage{float}}
\floatstyle{ruled}
\@ifundefined{c@chapter}{\newfloat{codelisting}{h}{lop}}{\newfloat{codelisting}{h}{lop}[chapter]}
\floatname{codelisting}{Listing}
\newcommand*\listoflistings{\listof{codelisting}{List of Listings}}
\makeatother
\makeatletter
\@ifpackageloaded{caption}{}{\usepackage{caption}}
\@ifpackageloaded{subcaption}{}{\usepackage{subcaption}}
\makeatother
\makeatletter
\@ifpackageloaded{tcolorbox}{}{\usepackage[many]{tcolorbox}}
\makeatother
\makeatletter
\@ifundefined{shadecolor}{\definecolor{shadecolor}{rgb}{.97, .97, .97}}
\makeatother
\makeatletter
\makeatother
\ifLuaTeX
  \usepackage{selnolig}  % disable illegal ligatures
\fi
\IfFileExists{bookmark.sty}{\usepackage{bookmark}}{\usepackage{hyperref}}
\IfFileExists{xurl.sty}{\usepackage{xurl}}{} % add URL line breaks if available
\urlstyle{same} % disable monospaced font for URLs
\hypersetup{
  pdftitle={InfiltrodiscR: an R package for infiltrometer data analysis and an experience for improving data reproducibility in a soil physics laboratory},
  pdfauthor={Carolina V. Giraldo; Sara E. Acevedo},
  colorlinks=true,
  linkcolor={blue},
  filecolor={Maroon},
  citecolor={Blue},
  urlcolor={Blue},
  pdfcreator={LaTeX via pandoc}}

\title{InfiltrodiscR: an R package for infiltrometer data analysis and
an experience for improving data reproducibility in a soil physics
laboratory}


\author[1,2]{Carolina V. Giraldo}
\author[1]{Sara E. Acevedo}

\affil[1]{Pontificia Universidad Católica de Chile}
\affil[2]{Centro de Desarrollo Urbano Sustentable (CEDEUS)}


\date{}
\begin{document}
\maketitle
\ifdefined\Shaded\renewenvironment{Shaded}{\begin{tcolorbox}[frame hidden, boxrule=0pt, breakable, sharp corners, interior hidden, borderline west={3pt}{0pt}{shadecolor}, enhanced]}{\end{tcolorbox}}\fi

\hypertarget{introduction}{%
\subsection{Introduction}\label{introduction}}

Many soil physics researchers are unacquainted about the functionalities
of the programing language R (Sousa et al. 2020). One main functionality
that differentiates it from spreadsheet-based programs is that R scripts
are text-based, making them easily shareable and reproducible, allowing
to replicate analyses. The challenge of achieving reproducibility
persists across various scientific disciplines, extending it to soil
science as well (Correndo et al. 2023). Also, researchers in soil
science, and almost every other field, are being pushed by funding
agencies and governmental institutions to increase transparency and
reproducibility of their work (Bond-Lamberty, Smith, and Bailey 2016).
Open, accessible, reusable, and reproducible soil hydrologic research
can have a significant positive impact on the scientific community and
broader society (Hall et al. 2022)

\hypertarget{motivation}{%
\subsection{Motivation}\label{motivation}}

In 2023, the authors of this work joined to a program lead by Open Life
Science; a community-oriented non-profit organisation that promotes
open, inclusive and equitable research (Haynes 2023). In addition, the
members of the laboratory already had knowledge of R but there was no
repository with common functions for the infiltrometer data analysis.
Based on the programming expertise of the laboratory members and the
need of adoption of open and reproducible science, the R package
InfiltrodiscR was developed (Acevedo and Giraldo 2023). The goal of
infiltrodiscR is to provide functions for the modeling of data derived
from the Minidisk Infiltrometer device.

\hypertarget{r-package-description}{%
\subsection{R package description}\label{r-package-description}}

The R package is currently hosted in GitHub. This web-based Git
repository hosting service is currently used by many scientists to work
in teams or collaborative projects (Bond-Lamberty, Smith, and Bailey
2016). Also, the code was deposited in Zenodo, a free service for
hosting data and software that offers long-term (20-year) storage and
integration with GitHub (Hall et al. 2022). The infiltrodiscR package
has a DOI so it can be used as reference in publications and clearly
define the software version used.

To install the R package, the users need to run to following lines:

\begin{Shaded}
\begin{Highlighting}[]
\CommentTok{\# install.packages("devtools")}
\NormalTok{devtools}\SpecialCharTok{::}\FunctionTok{install\_github}\NormalTok{(}\StringTok{"biofisicasuelos/infiltrodiscR"}\NormalTok{)}
\end{Highlighting}
\end{Shaded}

Data needed for running the functions are data stored in \textbf{.csv}
or \textbf{.xlsx} containing columns called as follows:

\begin{itemize}
\tightlist
\item
  texture: soil texture according to USDA: as.character() and lowercase,
  for example ``clay loam''.
\item
  suction: as.character() and lowercase, in this format: ``2cm''. Values
  allowed: ``0.5cm'',``1cm'',``2cm'',``3cm'',``4cm'',``5cm'',``6cm'',
  and ``7cm''.
\item
  volume: volume recorded in the infiltration measurements in mL,
  as.numeric().
\item
  time: time recorded in the infiltration measurements in seconds,
  as.numeric().
\end{itemize}

\hypertarget{main-functions}{%
\subsection{Main functions:}\label{main-functions}}

\textbf{\texttt{infiltration()}}

This function calculates cumulative infiltration and the square root of
time, using time and volume recorded based on the relationship described
by Philip (1957):

\[I = C_{1} t + C_{2} t^{0.5} \]

\textbf{\texttt{vg\_par}}

This function returns the parameter \emph{A}, \emph{no\_h} and
\emph{alpha} related to the van Genuchten parameters (Genuchten 1980),
from tabulated data calculated for a radius of 2.25 cm, including 12
soil texture classes and suctions from -0.5 cm to -7 cm. Table 1 show
selected data gathered from MeterGroup (2023) and Carsel and Parrish
(1988).

\begin{table}

\caption{Table 1. Selected data from the InfiltrodiscR package}
\centering
\begin{tabular}[t]{l|r|r|r|r|r}
\hline
texture & alpha & n/ho & 2cm & 4cm & 6cm\\
\hline
sand & 0.145 & 2.68 & 1.727908 & 0.8926213 & 0.4611201\\
\hline
loamy sand & 0.124 & 2.28 & 2.428600 & 1.8443631 & 1.4006735\\
\hline
sandy loam & 0.075 & 1.89 & 3.909913 & 3.9541483 & 3.9988836\\
\hline
loam & 0.036 & 1.56 & 6.267384 & 7.5304822 & 9.0481388\\
\hline
silt & 0.016 & 1.37 & 8.714378 & 9.8964325 & 11.2388254\\
\hline
silt loam & 0.020 & 1.41 & 7.929874 & 9.1856010 & 10.6401773\\
\hline
sandy clay loam & 0.059 & 1.48 & 4.242925 & 6.1530807 & 8.9231845\\
\hline
clay loam & 0.019 & 1.31 & 6.644845 & 7.8616094 & 9.3011807\\
\hline
silty clay loam & 0.010 & 1.23 & 8.511175 & 9.4110072 & 10.4059732\\
\hline
sandy clay & 0.027 & 1.23 & 4.089288 & 5.3640585 & 7.0362184\\
\hline
silty clay & 0.005 & 1.09 & 6.359575 & 6.7578953 & 7.1811641\\
\hline
clay & 0.008 & 1.09 & 4.300401 & 4.7393888 & 5.2231893\\
\hline
\end{tabular}
\end{table}

\textbf{\texttt{parameter\_A}}

This function returns the parameter \emph{A} calculated from the
equation based on the work developed by Zhang (1997), where the
parameters \emph{A}, \emph{no\_h} and \emph{alpha} determined previously
are input in the following equations described in MeterGroup (2023) and
Surda et al. (2019)

\[A = \frac{11.65(n^{0.1}-1)exp[2.92(n - 1.9)\alpha h_{0}]}{(\alpha r_{0})^{0.91}} ; n\geq 1.9 \]
\[A = \frac{11.65(n^{0.1}-1)exp[7.5(n - 1.9)\alpha h_{0}]}{(\alpha r_{0})^{0.91}} ; n < 1.9 \]

\newpage{}

\hypertarget{practical-example}{%
\subsection{Practical example}\label{practical-example}}

First, some dummy data about infiltration and soils is created. In this
case: ``soil\_a'' and ``soil\_b''.

\begin{Shaded}
\begin{Highlighting}[]
\NormalTok{infiltration\_data }\OtherTok{\textless{}{-}} \FunctionTok{tibble}\NormalTok{(}
  \AttributeTok{soil =} \FunctionTok{c}\NormalTok{(}\FunctionTok{rep}\NormalTok{(}\StringTok{"soil\_a"}\NormalTok{,}\DecValTok{11}\NormalTok{), }\FunctionTok{rep}\NormalTok{(}\StringTok{"soil\_b"}\NormalTok{,}\DecValTok{11}\NormalTok{)),}
  \AttributeTok{time =} \FunctionTok{c}\NormalTok{(}\DecValTok{0}\NormalTok{, }\DecValTok{30}\NormalTok{, }\DecValTok{60}\NormalTok{, }\DecValTok{90}\NormalTok{, }\DecValTok{120}\NormalTok{, }\DecValTok{150}\NormalTok{, }\DecValTok{180}\NormalTok{, }\DecValTok{210}\NormalTok{, }\DecValTok{240}\NormalTok{, }\DecValTok{270}\NormalTok{, }\DecValTok{300}\NormalTok{, }\CommentTok{\# seconds}
           \DecValTok{0}\NormalTok{, }\DecValTok{35}\NormalTok{, }\DecValTok{65}\NormalTok{, }\DecValTok{95}\NormalTok{, }\DecValTok{125}\NormalTok{, }\DecValTok{155}\NormalTok{, }\DecValTok{185}\NormalTok{, }\DecValTok{215}\NormalTok{, }\DecValTok{245}\NormalTok{, }\DecValTok{275}\NormalTok{, }\DecValTok{305}\NormalTok{),}
  \AttributeTok{volume =} \FunctionTok{c}\NormalTok{(}\DecValTok{95}\NormalTok{, }\DecValTok{89}\NormalTok{, }\DecValTok{86}\NormalTok{, }\DecValTok{83}\NormalTok{, }\DecValTok{80}\NormalTok{, }\DecValTok{77}\NormalTok{, }\DecValTok{74}\NormalTok{, }\DecValTok{73}\NormalTok{, }\DecValTok{71}\NormalTok{, }\DecValTok{69}\NormalTok{, }\DecValTok{67}\NormalTok{, }\CommentTok{\# mL}
             \DecValTok{83}\NormalTok{, }\DecValTok{77}\NormalTok{, }\DecValTok{64}\NormalTok{, }\DecValTok{61}\NormalTok{, }\DecValTok{58}\NormalTok{, }\DecValTok{45}\NormalTok{, }\DecValTok{42}\NormalTok{, }\DecValTok{35}\NormalTok{, }\DecValTok{29}\NormalTok{, }\DecValTok{17}\NormalTok{, }\DecValTok{15}\NormalTok{)}
\NormalTok{)}

\NormalTok{soil\_data }\OtherTok{\textless{}{-}} \FunctionTok{tibble}\NormalTok{(}\AttributeTok{soil =} \FunctionTok{c}\NormalTok{(}\StringTok{"soil\_a"}\NormalTok{, }\StringTok{"soil\_b"}\NormalTok{),}
                    \AttributeTok{texture =} \FunctionTok{c}\NormalTok{(}\StringTok{"sandy loam"}\NormalTok{, }\StringTok{"clay loam"}\NormalTok{), }\CommentTok{\#USDA}
                    \AttributeTok{suction =} \FunctionTok{c}\NormalTok{(}\StringTok{"4cm"}\NormalTok{,}\StringTok{"2cm"}\NormalTok{),}
                    \AttributeTok{om\_content =} \FunctionTok{c}\NormalTok{(}\DecValTok{1}\NormalTok{,}\DecValTok{10}\NormalTok{))}

\FunctionTok{head}\NormalTok{(infiltration\_data,}\DecValTok{4}\NormalTok{) }\CommentTok{\# check the infiltration data}
\end{Highlighting}
\end{Shaded}

\begin{verbatim}
# A tibble: 4 x 3
  soil    time volume
  <chr>  <dbl>  <dbl>
1 soil_a     0     95
2 soil_a    30     89
3 soil_a    60     86
4 soil_a    90     83
\end{verbatim}

\begin{Shaded}
\begin{Highlighting}[]
\NormalTok{soil\_data }\CommentTok{\# check the soil data}
\end{Highlighting}
\end{Shaded}

\begin{verbatim}
# A tibble: 2 x 4
  soil   texture    suction om_content
  <chr>  <chr>      <chr>        <dbl>
1 soil_a sandy loam 4cm              1
2 soil_b clay loam  2cm             10
\end{verbatim}

Then, using the function \textbf{\texttt{infiltration()}} the cumulative
infiltration and the square root of time are calculated. Notice that the
package was coded tidy-oriented (tidyverse package is required). Also,
it is recommended to use nested tibbles for data manipulation.

\begin{Shaded}
\begin{Highlighting}[]
\NormalTok{infilt\_cum\_sqrt }\OtherTok{\textless{}{-}}
\NormalTok{infiltration\_data }\SpecialCharTok{\%\textgreater{}\%} 
\FunctionTok{group\_by}\NormalTok{(soil) }\SpecialCharTok{\%\textgreater{}\%} \CommentTok{\# grouped calculation by soil}
\FunctionTok{nest}\NormalTok{() }\SpecialCharTok{\%\textgreater{}\%} 
\FunctionTok{mutate}\NormalTok{(}\AttributeTok{data =} \FunctionTok{map}\NormalTok{(data, }\SpecialCharTok{\textasciitilde{}}\NormalTok{ infiltrodiscR}\SpecialCharTok{::}\FunctionTok{infiltration}\NormalTok{(.), }\AttributeTok{data =}\NormalTok{ .x)) }

\NormalTok{infilt\_cum\_sqrt }\CommentTok{\# nested tibble}
\end{Highlighting}
\end{Shaded}

\begin{verbatim}
# A tibble: 2 x 2
# Groups:   soil [2]
  soil   data             
  <chr>  <list>           
1 soil_a <tibble [11 x 5]>
2 soil_b <tibble [11 x 5]>
\end{verbatim}

The nested tibble has the infiltration calculation for each soil. For
details of \textbf{\texttt{infilt\_cum\_sqrt}}, the
\textbf{\texttt{unnest()}} function can be used

\begin{Shaded}
\begin{Highlighting}[]
\NormalTok{infilt\_cum\_sqrt }\SpecialCharTok{\%\textgreater{}\%} 
  \FunctionTok{unnest}\NormalTok{(data)}
\end{Highlighting}
\end{Shaded}

\begin{verbatim}
# A tibble: 22 x 6
# Groups:   soil [2]
   soil    time volume sqrt_time volume_infiltrated infiltration
   <chr>  <dbl>  <dbl>     <dbl>              <dbl>        <dbl>
 1 soil_a     0     95      0                     0         0   
 2 soil_a    30     89      5.48                  6         0.38
 3 soil_a    60     86      7.75                  9         0.57
 4 soil_a    90     83      9.49                 12         0.75
 5 soil_a   120     80     11.0                  15         0.94
 6 soil_a   150     77     12.2                  18         1.13
 7 soil_a   180     74     13.4                  21         1.32
 8 soil_a   210     73     14.5                  22         1.38
 9 soil_a   240     71     15.5                  24         1.51
10 soil_a   270     69     16.4                  26         1.63
# i 12 more rows
\end{verbatim}

Now the soil data can be joined to the infiltration data and the Van
Genuchten parameters can be obtained. It is mandatory to have a column
called \textbf{\texttt{texture}} and another \textbf{\texttt{suction}}

\begin{Shaded}
\begin{Highlighting}[]
\NormalTok{infilt\_cum\_sqrt }\SpecialCharTok{\%\textgreater{}\%} 
  \FunctionTok{left\_join}\NormalTok{(soil\_data) }\SpecialCharTok{\%\textgreater{}\%} 
\NormalTok{  infiltrodiscR}\SpecialCharTok{::}\FunctionTok{vg\_par}\NormalTok{()}
\end{Highlighting}
\end{Shaded}

\begin{verbatim}
Joining with `by = join_by(soil)`
Joining with `by = join_by(texture)`
\end{verbatim}

\begin{verbatim}
# A tibble: 2 x 8
# Groups:   soil [2]
  soil   data              texture    suction om_content alpha  n_ho value_A
  <chr>  <list>            <chr>      <chr>        <dbl> <dbl> <dbl>   <dbl>
1 soil_a <tibble [11 x 5]> sandy loam 4cm              1 0.075  1.89    3.95
2 soil_b <tibble [11 x 5]> clay loam  2cm             10 0.019  1.31    6.64
\end{verbatim}

The hydraulic conductivity of the soil K at a specific suction is
calculated as: \[K_{(h)} = \frac{C_{1}}{A}\] Parameter
C\textsubscript{1} is calculated fitting a polynomial function of the
second degree (y = ax2+b), where a is parameter C\textsubscript{1}, x is
the square root of time and y is the cumulative infiltration calculated
previously. For this step, we use the package broom and base R. The
column estimate corresponds to the parameter C\textsubscript{1}.

\begin{Shaded}
\begin{Highlighting}[]
\NormalTok{processed\_data }\OtherTok{\textless{}{-}} 
\NormalTok{infilt\_cum\_sqrt }\SpecialCharTok{\%\textgreater{}\%} 
  \FunctionTok{left\_join}\NormalTok{(soil\_data) }\SpecialCharTok{\%\textgreater{}\%} 
\NormalTok{  infiltrodiscR}\SpecialCharTok{::}\FunctionTok{vg\_par}\NormalTok{() }\SpecialCharTok{\%\textgreater{}\%} 
    \FunctionTok{mutate}\NormalTok{(}
    \AttributeTok{fit =} \FunctionTok{map}\NormalTok{(data,}
              \SpecialCharTok{\textasciitilde{}} \FunctionTok{lm}\NormalTok{(infiltration }\SpecialCharTok{\textasciitilde{}} \FunctionTok{poly}\NormalTok{(sqrt\_time, }\DecValTok{2}\NormalTok{, }\AttributeTok{raw =} \ConstantTok{TRUE}\NormalTok{),}
                   \AttributeTok{data =}\NormalTok{ .x)), }\CommentTok{\#polynomial function}
    \AttributeTok{tidied =} \FunctionTok{map}\NormalTok{(fit, broom}\SpecialCharTok{::}\NormalTok{tidy) }\CommentTok{\#coefficients}
\NormalTok{  ) }\SpecialCharTok{\%\textgreater{}\%} 
  \FunctionTok{unnest}\NormalTok{(tidied) }\SpecialCharTok{\%\textgreater{}\%} 
\FunctionTok{filter}\NormalTok{(term }\SpecialCharTok{==} \StringTok{"poly(sqrt\_time, 2, raw = TRUE)2"}\NormalTok{) }\SpecialCharTok{\%\textgreater{}\%} \CommentTok{\#slope}
\FunctionTok{rename}\NormalTok{(}\AttributeTok{C1 =}\NormalTok{ estimate) }
\end{Highlighting}
\end{Shaded}

\begin{verbatim}
Joining with `by = join_by(soil)`
Joining with `by = join_by(texture)`
\end{verbatim}

\newpage{}

Finally, the hydraulic conductivity of the soil K is calculating using
the parameter C\textsubscript{1} and A. If seconds and mL were used as
inputs for infiltration data, the units of K are cm/s.

\begin{Shaded}
\begin{Highlighting}[]
\NormalTok{processed\_data }\SpecialCharTok{\%\textgreater{}\%} 
\NormalTok{  infiltrodiscR}\SpecialCharTok{::}\FunctionTok{parameter\_A}\NormalTok{() }\SpecialCharTok{\%\textgreater{}\%} 
  \FunctionTok{mutate}\NormalTok{(}\AttributeTok{K\_h =}\NormalTok{ C1 }\SpecialCharTok{/}\NormalTok{ parameter\_A) }\SpecialCharTok{\%\textgreater{}\%} 
  \FunctionTok{select}\NormalTok{(soil, texture, suction, K\_h)}
\end{Highlighting}
\end{Shaded}

\begin{verbatim}
# A tibble: 2 x 4
# Groups:   soil [2]
  soil   texture    suction      K_h
  <chr>  <chr>      <chr>      <dbl>
1 soil_a sandy loam 4cm     0.000638
2 soil_b clay loam  2cm     0.00212 
\end{verbatim}

Using this tidy-oriented approach, it is simple to complement the
functions presented with plotting.

\begin{Shaded}
\begin{Highlighting}[]
\NormalTok{infiltration\_plot }\OtherTok{\textless{}{-}} 
\NormalTok{infilt\_cum\_sqrt }\SpecialCharTok{\%\textgreater{}\%} 
  \FunctionTok{left\_join}\NormalTok{(soil\_data) }\SpecialCharTok{\%\textgreater{}\%} 
  \FunctionTok{mutate}\NormalTok{(}\AttributeTok{plot =} \FunctionTok{map2}\NormalTok{(}
\NormalTok{    data, soil, }
    \SpecialCharTok{\textasciitilde{}} \FunctionTok{ggplot}\NormalTok{(}\AttributeTok{data =}\NormalTok{ .x, }\FunctionTok{aes}\NormalTok{(}\AttributeTok{x =}\NormalTok{ sqrt\_time, }\AttributeTok{y =}\NormalTok{ infiltration)) }\SpecialCharTok{+}
      \FunctionTok{ggtitle}\NormalTok{(glue}\SpecialCharTok{::}\FunctionTok{glue}\NormalTok{(}\StringTok{"Soil : \{soil\}}
\StringTok{                   Suction : \{suction\}"}\NormalTok{)) }\SpecialCharTok{+}
      \FunctionTok{stat\_smooth}\NormalTok{(}\AttributeTok{method=}\StringTok{\textquotesingle{}lm\textquotesingle{}}\NormalTok{, }\AttributeTok{formula =}\NormalTok{ y}\SpecialCharTok{\textasciitilde{}}\FunctionTok{poly}\NormalTok{(x,}\DecValTok{2}\NormalTok{)) }\SpecialCharTok{+}
      \FunctionTok{geom\_point}\NormalTok{() }\SpecialCharTok{+}
      \FunctionTok{theme\_bw}\NormalTok{()))}
\end{Highlighting}
\end{Shaded}

\begin{verbatim}
Joining with `by = join_by(soil)`
\end{verbatim}

\begin{Shaded}
\begin{Highlighting}[]
\NormalTok{patchwork}\SpecialCharTok{::}\FunctionTok{wrap\_plots}\NormalTok{(infiltration\_plot}\SpecialCharTok{$}\NormalTok{plot, }\AttributeTok{ncol =} \DecValTok{2}\NormalTok{)  }
\end{Highlighting}
\end{Shaded}

\begin{figure}[H]

{\centering \includegraphics{infiltrodiscR_paper_files/figure-pdf/unnamed-chunk-9-1.pdf}

}

\end{figure}

\hypertarget{conclusions-and-future-work}{%
\subsection{Conclusions and future
work}\label{conclusions-and-future-work}}

The learning curve in R programming and open science practices is not
the same for every researcher, nor is it a dedicated line of research in
graduate programs dedicated to soil physics. Therefore, this experience
in creating an R package homogenizing the data analysis methodology in a
laboratory shows that if there is interest in developing this approach,
further advances in collaboration and reproducibility can be made.

\hypertarget{acknowledgements}{%
\subsection{Acknowledgements}\label{acknowledgements}}

The authors thank the OLS team for their time and dedication in
motivating researchers to adopt open software practices. Sara Acevedo
thanks the research support provided by CEDEUS (ANID/FONDAP 1522A0002)
and the financial support from Postdoctorado Ingeniería UC 2023.

\hypertarget{references}{%
\subsection*{References}\label{references}}
\addcontentsline{toc}{subsection}{References}

\hypertarget{refs}{}
\begin{CSLReferences}{1}{0}
\leavevmode\vadjust pre{\hypertarget{ref-https:ux2fux2fdoi.orgux2f10.5281ux2fzenodo.8001894}{}}%
Acevedo, Sara, and Carolina Giraldo. 2023. {``infiltrodiscR: R
Package.''} Zenodo. \url{https://doi.org/10.5281/ZENODO.8001894}.

\leavevmode\vadjust pre{\hypertarget{ref-BondLamberty2016}{}}%
Bond-Lamberty, Ben, A Peyton Smith, and Vanessa Bailey. 2016. {``Running
an Open Experiment: Transparency and Reproducibility in Soil and
Ecosystem Science.''} \emph{Environmental Research Letters} 11 (8):
084004. \url{https://doi.org/10.1088/1748-9326/11/8/084004}.

\leavevmode\vadjust pre{\hypertarget{ref-Carsel1988}{}}%
Carsel, Robert F., and Rudolph S. Parrish. 1988. {``Developing Joint
Probability Distributions of Soil Water Retention Characteristics.''}
\emph{Water Resources Research} 24 (5): 755--69.
\url{https://doi.org/10.1029/wr024i005p00755}.

\leavevmode\vadjust pre{\hypertarget{ref-CORRENDO2023101275}{}}%
Correndo, Adrian A., Austin Pearce, Carl H. Bolster, John T. Spargo,
Deanna Osmond, and Ignacio A. Ciampitti. 2023. {``The Soiltestcorr r
Package: An Accessible Framework for Reproducible Correlation Analysis
of Crop Yield and Soil Test Data.''} \emph{SoftwareX} 21: 101275.
https://doi.org/\url{https://doi.org/10.1016/j.softx.2022.101275}.

\leavevmode\vadjust pre{\hypertarget{ref-vanGenuchten1980}{}}%
Genuchten, M. Th. van. 1980. {``A Closed-Form Equation for Predicting
the Hydraulic Conductivity of Unsaturated Soils.''} \emph{Soil Science
Society of America Journal} 44 (5): 892--98.
\url{https://doi.org/10.2136/sssaj1980.03615995004400050002x}.

\leavevmode\vadjust pre{\hypertarget{ref-Hall2022}{}}%
Hall, Caitlyn A., Sheila M. Saia, Andrea L. Popp, Nilay Dogulu,
Stanislaus J. Schymanski, Niels Drost, Tim van Emmerik, and Rolf Hut.
2022. {``A Hydrologist{\textquotesingle}s Guide to Open Science.''}
\emph{Hydrology and Earth System Sciences} 26 (3): 647--64.
\url{https://doi.org/10.5194/hess-26-647-2022}.

\leavevmode\vadjust pre{\hypertarget{ref-Haynes2023}{}}%
Haynes, Sam. 2023. {``{OLS}: Capacity Building in Open Science with a
Peer-Led, Global, and Diverse Community.''} \emph{Edinburgh Open
Research}, June. \url{https://doi.org/10.2218/eor.2023.8120}.

\leavevmode\vadjust pre{\hypertarget{ref-METER}{}}%
MeterGroup. 2023. \emph{Mini Disk Infiltrometer - Meter Group}.
\url{http://publications.metergroup.com/Manuals/20421_Mini_Disk_Manual_Web.pdf}.

\leavevmode\vadjust pre{\hypertarget{ref-Philip1957THETO}{}}%
Philip, J. R. 1957. {``THE THEORY OF INFILTRATION: 4. SORPTIVITY AND
ALGEBRAIC INFILTRATION EQUATIONS.''} \emph{Soil Science} 84: 257--64.

\leavevmode\vadjust pre{\hypertarget{ref-10.1016ux2fj.compag.2019.105077}{}}%
Sousa, Decı́ola Fernandes de, Sueli Rodrigues, Herdjania Veras de Lima,
and Lorena Torres Chagas. 2020. {``R Software Packages as a Tool for
Evaluating Soil Physical and Hydraulic Properties.''} \emph{Comput.
Electron. Agric.} 168 (C).
\url{https://doi.org/10.1016/j.compag.2019.105077}.

\leavevmode\vadjust pre{\hypertarget{ref-Surda2019}{}}%
Surda, Peter, Justina Vitkova, Katarina Brezianska, and Lubomir Lichner.
2019. {``Influence of the Infiltration Disk Radius on Determination of
Unsaturated Hydraulic Conductivity of Non-Structural Sandy Soil.''}
\emph{{IOP} Conference Series: Earth and Environmental Science} 221
(March): 012024. \url{https://doi.org/10.1088/1755-1315/221/1/012024}.

\leavevmode\vadjust pre{\hypertarget{ref-Zhang1997}{}}%
Zhang, Renduo. 1997. {``Infiltration Models for the Disk
Infiltrometer.''} \emph{Soil Science Society of America Journal} 61 (6):
1597--1603.
\url{https://doi.org/10.2136/sssaj1997.03615995006100060008x}.

\end{CSLReferences}



\end{document}
